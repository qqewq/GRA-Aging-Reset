\documentclass[11pt,a4paper]{article}

\usepackage[utf8]{inputenc}
\usepackage[T2A]{fontenc}
\usepackage[russian,english]{babel}
\usepackage{geometry}
\geometry{margin=2.5cm}
\usepackage{hyperref}
\usepackage{amsmath,amssymb}
\usepackage{graphicx}
\usepackage{authblk}

\title{%
GRA-Aging-Reset:\\
Многоуровневая GRA-Мета-обнулёнка и эпигенетическая теория старения
}

\author[1]{Твоё Имя}
\affil[1]{Независимый исследователь}

\date{Февраль 2026}

\begin{document}
\maketitle

\begin{abstract}
В работе предлагается многоуровневая GRA-модель старения, связывающая клеточные, тканевые, органные и системные уровни с эпигенетическим мета-уровнем, на котором старение рассматривается как рост информационной ``пены''. Обсуждаются эпигенетические часы как числовая мера этой пены, роль сенесцентных клеток и сенолитиков, системных сигналов (метаболизм, гормоны, иммунитет), а также возможные стратегии многоуровневого ``обнуления'' старения, от консервативных вмешательств в образ жизни до экспериментальных подходов (сенолитики, rapalogs, эпигенетическое перепрограммирование). Предлагается качественный роадмап 2030–2050 и обсуждаются риски и этические ограничения.
\end{abstract}

\section{Введение}
Кратко: зачем модель GRA, почему старение удобно понимать как многоуровневый процесс с единой информационной причиной на эпигенетическом мета-уровне, и как это стыкуется с интуицией Маска и эпигенетической теорией Синклера.

\section{Многоуровневая GRA-модель}
Здесь можно вклеить/переписать содержимое \texttt{01\_GRA\_multilevel\_formalism.md} и \texttt{02\_GRA\_applied\_to\_aging.md} в виде текста и формул.

\section{Эпигенетическая информационная теория старения}
Здесь конденсируешь \texttt{01\_epigenetic\_information\_theory.md} и обзор эпигенетических часов (\texttt{02\_epigenetic\_clocks\_survey.md}), со ссылками на ключевые работы.

\section{Клеточная сенесценция и сенолитики}
Кратко пересказываешь \texttt{03\_senescence\_senolytics.md}.

\section{Системные сигналы: метаболизм, гормоны, иммунитет}
Сжатая версия \texttt{04\_systemic\_signals\_hormones\_immune.md}.

\section{Практическая многоуровневая стратегия}
Суммарный пересказ \texttt{01\_strategy\_overview.md}, \texttt{02\_lifestyle\_protocol.md}, \texttt{03\_drug\_supplement\_landscape.md}.

\section{Инструменты и экспериментальный роадмап}
Кратко: \texttt{epigenetic\_clock\_tools.md}, \texttt{04\_experimental\_roadmap\_lab.md}.

\section{Риски, этика и роадмап 2030–2050}
Основано на \texttt{01\_risks\_ethics.md}, \texttt{02\_roadmap\_2030\_2050.md}.

\section{Заключение}
Пара абзацев о том, что GRA-модель даёт общий язык для связывания биологии старения, практических протоколов и будущих технологий обнуления.

\bibliographystyle{plain}
\bibliography{references}

\end{document}
